\documentclass[a4paper, 11pt]{article}

\usepackage[french]{babel}
\usepackage[utf8]{inputenc}  
\usepackage[T1]{fontenc}
\usepackage[left=3cm,right=3cm,top=3cm,bottom=3cm]{geometry}
\usepackage{graphicx}
\usepackage{parskip}
\usepackage{titling}

\setlength{\droptitle}{-6em}   % This is your set screw

\title{
	\noindent\rule{\linewidth}{0.4pt}
	\huge{Compte-Rendu Réunion 4\\}
	\medskip
	\Large{PAM --- Auto-Oscillations des Instruments de Musique}
	\noindent\rule{\linewidth}{1pt}
}
\author{Durand, Le, Salvador, Verrier}
\date{10 Février 2021}

\begin{document}

\maketitle


\section{Récapitulatif de la réunion}
\subsection{Cartographie des régimes}
\begin{itemize}
	\item Présentation des résultats avec le descripteur périodique/quasi-périodique et simulation avec paramètres variables
	\item Explications des schémas numériques Euler explicite/implicite et lien avec les algorithmes d'intégrations (méthode des rectangles)
\end{itemize}
\subsection{Approche ligne à retard}
\begin{itemize}
	\item Présentation des résultats actuels avec l'approche acoustique de la fonction de réflexion (approximation BF)
	\item Mise au point sur le modèle général
\end{itemize}

\section{Objectifs pour la prochaine réunion}
\begin{itemize}
	\item \textbf{Approche modale}
	\begin{itemize}
		\item Implémenter l'approche modale compatible temps réel sous Matlab avec Euelr explicite
		\item Implémenter sous Max/MSP
	\end{itemize}
	\item \textbf{Approche ligne à retard}
	\begin{itemize}
		\item Utilisation d'un filtre passe-bas pour la fonction de réflexion
		\item Implémenter l'approche modale sous Max/MSP
	\end{itemize}
	\item \textbf{Rendus}
	\begin{itemize}
		\item Rédiger le rapport
		\item Commencer la présentation
	\end{itemize}
\end{itemize}


\end{document}