\documentclass[a4paper, 11pt]{article}

\usepackage[french]{babel}
\usepackage[utf8]{inputenc}  
\usepackage[T1]{fontenc}
\usepackage[left=3cm,right=3cm,top=3cm,bottom=3cm]{geometry}
\usepackage{graphicx}
\usepackage{parskip}
\usepackage{titling}

\setlength{\droptitle}{-6em}   % This is your set screw

\title{
	\noindent\rule{\linewidth}{0.4pt}
	\huge{Compte-Rendu Réunion 2\\}
	\medskip
	\Large{PAM --- Auto-Oscillations des Instruments de Musique}
	\noindent\rule{\linewidth}{1pt}
}
\author{Durand, Le, Salvador, Verrier}
\date{27 Janvier 2021}

\begin{document}

\maketitle


\section{Récapitulatif de la réunion}
	\subsection{Approche modale}
	\begin{itemize}
		\item Présentation des résultats de simulation
		\begin{itemize}
			\item Implémentation du système avec 5 modes, basé sur le TP Vents 
			\item Le code doit être parfois mieux écrit pour être compréhensible
		\end{itemize}
		\item Temps réel sous Simulink
		\begin{itemize}
			\item Implémentation du système avec 1 mode
			\item Il reste encore beaucoup de problèmes, en particulier dans la sortie sonore
		\end{itemize}
	\end{itemize}
		
	\subsection{Approche ligne à retard}
	\begin{itemize}
		\item Discussion sur la forme de la fonction de réflexion 
		\begin{itemize}
			\item Rôle du signe "-"
			\item Maximum de la fonction de réflexion et pertes (rayonnement, ...)
		\end{itemize}
		\item Discussion sur le modèle de lignes à retard 
		\begin{itemize}
			\item Rôle des différents éléments (carte itérée, retard, convolution par $r(t)$, ...)
			\item Implémentation informatique en pratique (utilisation d'un buffer, ...)
		\end{itemize}
	\end{itemize}

\section{Objectifs pour la prochaine réunion}
\begin{itemize}
	\item Exprimer la fonction de réflexion en connaissant l'impédance du tube et l'impédance de rayonnement, puis analyse de la fonction (Diagramme de Bode..)
	\item Prouver que le lien entre $(p^{+}, p^{-})$ et $(p,u)$ est une rotation de 45°
	\item Comprendre, formaliser et implémenter l'approche ligne à retard
	\item Comprendre quel solveur on utilise (\texttt{dopri45} ou autres) dans l'approche modale
	\item Avoir une meilleure ébauche de transposition vers le temps réel pour l'approche modale		
	\item (Après le cours de vendredi) Explorer les notions de cartographie des régimes d'oscillations 
	\item (Lors de la prochaine réunion) De nouveau parler de la paramétrisation de la fonction de réflexion
\end{itemize}

\end{document}