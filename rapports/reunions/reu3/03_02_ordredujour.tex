\documentclass[a4paper, 11pt]{article}

\usepackage[french]{babel}
\usepackage[utf8]{inputenc}  
\usepackage[T1]{fontenc}
\usepackage[left=3cm,right=3cm,top=3cm,bottom=3cm]{geometry}
\usepackage{graphicx}
\usepackage{parskip}
\usepackage{titling}

\setlength{\droptitle}{-6em}   % This is your set screw

\title{
	\noindent\rule{\linewidth}{0.4pt}
	\huge{Ordre du Jour Réunion 3\\}
	\medskip
	\Large{PAM --- Auto-Oscillations des Instruments de Musique}
	\noindent\rule{\linewidth}{1pt}
}
\author{Durand, Le, Salvador, Verrier}
\date{3 Février 2021}

\begin{document}

\maketitle


\begin{enumerate}
	\item \textbf{Approche modale}
	\begin{enumerate}
		\item Présentation des cartes calculées
		\item Simulation avec paramètres de jeu variables
	\end{enumerate}
	
	\item \textbf{Approche ligne à retard}
	\begin{enumerate}
		\item Résultats des exercices théoriques (rotation 45°, expression de la fonction de réflexion)
		\item Présentation de l'implémentation
	\end{enumerate}

	\item \textbf{Temps réel}
	\begin{enumerate}
		\item Aspects pratiques - fonctionnement de la tablette, externals Max utilisés...
	\end{enumerate}
		
\end{enumerate}

\end{document}