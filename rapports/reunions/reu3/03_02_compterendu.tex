\documentclass[a4paper, 11pt]{article}

\usepackage[french]{babel}
\usepackage[utf8]{inputenc}  
\usepackage[T1]{fontenc}
\usepackage[left=3cm,right=3cm,top=3cm,bottom=3cm]{geometry}
\usepackage{graphicx}
\usepackage{parskip}
\usepackage{titling}

\setlength{\droptitle}{-6em}   % This is your set screw

\title{
	\noindent\rule{\linewidth}{0.4pt}
	\huge{Compte-Rendu Réunion 3\\}
	\medskip
	\Large{PAM --- Auto-Oscillations des Instruments de Musique}
	\noindent\rule{\linewidth}{1pt}
}
\author{Durand, Le, Salvador, Verrier}
\date{3 Février 2021}

\begin{document}

\maketitle


\section{Récapitulatif de la réunion}
\subsection{Cartographie des régimes}
\begin{itemize}
	\item Présentation de la carte présence/absence de régime oscillant et de stabilité numérique
	\item Présentation de la tentative de carte de régime quasi-périodique / périodique et discussion des paramètres
\end{itemize}
\subsection{Approche ligne à retard}
\begin{itemize}
	\item Mise au point sur l'implémentation actuelle de l'approche
	\item Discussion sur la forme de $R(\omega)$ et des hypothèses basses fréquences probablement pas respectées pour l'expression de $Z_R$ 
\end{itemize}

\section{Objectifs pour la prochaine réunion}
\begin{itemize}
	\item \textbf{Cartographie des régimes}
	\begin{itemize}
		\item Se fixer des descripteurs (MIR-Toolbox, à la main...)
		\item Se fixer des paramètres pour les axes de la carte ($\gamma$, $\zeta$, $\ell$, inharmonicité...)
		\item Résoudre le problème de l'instabilité numérique
	\end{itemize}
	\item \textbf{Approche ligne à retard}
	
	\item \textbf{Temps réel \& autres}
	\begin{itemize}
		\item Se renseigner sur la compatibilité de la tablette avec des externals MaxMSP sous Windows
		\item Commencer à rédiger le rapport
	\end{itemize}
\end{itemize}


\end{document}