\documentclass[a4paper, 11pt]{article}

\usepackage[french]{babel}
\usepackage[utf8]{inputenc}  
\usepackage[T1]{fontenc}
\usepackage[left=3cm,right=3cm,top=3cm,bottom=3cm]{geometry}
\usepackage{graphicx}
\usepackage{parskip}
\usepackage{titling}

\setlength{\droptitle}{-6em}   % This is your set screw

\title{
	\noindent\rule{\linewidth}{0.4pt}
	\huge{Compte-Rendu Réunion 5\\}
	\medskip
	\Large{PAM --- Auto-Oscillations des Instruments de Musique}
	\noindent\rule{\linewidth}{1pt}
}
\author{Durand, Le, Salvador, Verrier}
\date{17 Février 2021}

\begin{document}

\maketitle


\section{Récapitulatif de la réunion}
\subsection{Présentation de la simulation en temps réel}
\begin{itemize}
	\item Approche modale
	\item Approche ligne de retard
\end{itemize}


\section{Objectifs pour la présentation}
\begin{itemize}
	\item \textbf{Organisation du plan}
	\begin{itemize}
		\item Penser au jury et aux autres étudiants
		\item Sinopsis: differentioer entre la partie acoustique, la partie traitement de signal et la partie informatique
	\end{itemize}
	\item \textbf{Demo en temps réel}
	\begin{itemize}
		\item Créer une nouvelle cartographie par superposition d'autres cartographies
		\item Penser à faire des vidéos de securité
	\end{itemize}

\end{itemize}


\end{document}
