\documentclass{article}
\usepackage[utf8]{inputenc}
\usepackage[sc]{mathpazo}

\usepackage[T1]{fontenc}
\usepackage{lmodern}
% \renewcommand{\familydefault}{\sfdefault}

%\usepackage[pdftex,
%            pdfauthor={LE Dinh-Viet-Toan},
%           pdftitle={Title},
%            pdfcreator={pdflatex},
%            hidelinks]{hyperref}

\usepackage{ragged2e}
\usepackage[a4paper, right=2.5cm,left=2.5cm,top=2.5cm,bottom=2.5cm]{geometry}

\usepackage{mathtools} % Basic maths
\usepackage{mathabx}
\usepackage{indentfirst} % Indentation at the beginning of a section
\usepackage{textcomp} % Extra characters
\usepackage{todonotes} 
\usepackage{multicol} % Multi columns
\usepackage{cancel} % Draw line to cancel a term

\usepackage[export]{adjustbox}
\usepackage{subcaption}

% Graphics
\usepackage{tikz}
\usetikzlibrary{shapes,arrows,shadings,patterns}
\usepackage{natbib}
\usepackage{graphicx,wrapfig,lipsum}

% Disable number in section
\let\oldsection\section
\renewcommand{\section}[1]{\oldsection*{#1}}
\let\oldsubsection\subsection
\renewcommand{\subsection}[1]{\oldsubsection*{\hspace{1em}#1}}

% Extra commands
\newcommand\norm[1]{\left\lVert#1\right\rVert}
\newcommand{\pplus}{p^{+}}
\newcommand{\pmoins}{p^{-}}
\newcommand{\sqrttwoovtwo}{\frac{\sqrt{2}}{2}}
\newcommand{\powinv}{^{-1}}



% Math operators
\DeclareMathOperator\Imspace{Im}

\providecommand{\dividespace}{\vspace{1.6em}}

\title{Calcul de la relation entrée/sortie associée à la fonction de réflexion }
\author{}
\date{}



\begin{document}

\maketitle 
Nous cherchons à exprimer l'échantillon $p^-(n)$ de l'onde retour en fonction des échantillons précédents et de l'onde aller $p^+(n)$. Pour cela nous discrétisons l'équation différentielle associée à la fonction de réflexion $R(\omega)$ 
Nous avons exprimé la réponse en fréquence de la fonction de réflexion :

\begin{align}
R(\omega) & = \frac{P^+(\omega)}{P^-(\omega)}\\
& = \frac{Z_r(\omega)-Z_c}{Z_r(\omega)+Z_c}
\end{align}
Avec 
\begin{align*}
z_c & =\rho c/S\\
Z_r & = Z_c\left(0.25(ka)^2+0.6133jka\right)
\end{align*}
$P^+$ désigne la transformée de Fourrier de l'onde aller, $P^-$ celle de l'onde retour. Dans la suite, on note $Q^{\pm}(\omega)=P^{\pm}(\omega)Z_c$, de transformée inverse $q_{\pm}(t)=p^{\pm}(t)Z_c$, et $z_r(\omega)=Z_cZ_r(\omega)$\\
On en déduit, en regroupant les ondes retour dans le membre de gauche et les ondes allers dans le membre de droite :
\begin{align*}
Q^-\left[z_r(\omega)+1\right]&=Q^+\left[z_r(\omega)-1\right]\\
\leftrightarrow 0.25 (a/c)^2\omega^2Q^-(\omega)+0.6133(a/c)j\omega Q^-(\omega) +Q^-(\omega) &=0.25(a/c)^2\omega^2Q^+(\omega)+0.6133j\omega Q^+(\omega)-Q^+(\omega)
\end{align*}
On obtient une équation différentielle reliant $q^+(t)$ et $q^-(t)$ :
\begin{equation}
-0.25(a/c)^2\frac{\mathrm{d}^2p^-(t)}{\mathrm{d}t^2}+0.6133(a/c)\frac{\mathrm{d}p^-(t)}{\mathrm{d}t}+p^-(t)=-0.25(a/c)^2\frac{\mathrm{d}^2p^+(t)}{\mathrm{d}t^2}+0.6133(a/c)\frac{\mathrm{d}p^+(t)}{\mathrm{d}t}-p^+(t)
\label{eqdiff}
\end{equation}
On discrétise \ref{eqdiff} en utilisant un schéma en différences finies à gauche (bon terme ??) à l'ordre 2:
\begin{align*}
\frac{\mathrm{d}f(n)}{\mathrm{d}t} & =(3/2)f(n)-2f(n-1)+(1/2)f(n-2)\\
\frac{\mathrm{d}^2f(n)}{\mathrm{d}t} & = 2f(n)-5f(n-1)+4f(n-2)1f(n-3)
\end{align*}
On obtient alors une expression de $q^-(n)$:
\begin{align}
q^-(n)=A^{-1}\left[E\left(q^+(n)\right)-a_1q^-(n-1)-a_2q^-(n-2)-a_3q^-(n-3)\right]
\end{align}
Avec
\begin{align*}
E\left(q^+(n)\right) & =-0.25(a/c)^2\frac{\mathrm{d}^2p^+(n)}{\mathrm{d}t^2}+0.6133(a/c)\frac{\mathrm{d}p^+(n)}{\mathrm{d}t}-p^+(n)\\
a_1 & =-2\times0.6133\frac{a}{c}+5\times0.25\left(\frac{a}{c}\right)^2\\
a_2 & =0.5\times0.6133\frac{a}{c}+4\times0.25\left(\frac{a}{c}\right)^2\\
a_3 & =-1\times0.25\left(\frac{a}{c}\right)^2
\end{align*}


\end{document}


